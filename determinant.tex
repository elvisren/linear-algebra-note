\chapter{Determinant}


\section{Definition}

Determinant is a function from $A$ to $F$. The most used one is 
\begin{equation}
    \determinate{\begin{pmatrix}
        a & b \\
        c & d
    \end{pmatrix}} = ad - bc
\end{equation}


Proper definition of determinant is a little complex. Define the cofactor of the entry of $A$ in row $i$ and column $j$ as $\cofactor{A} = (-1)^{i+j} \determinate{\tilde{A}_{ij}}$, $\tilde{A}$ is a $(n-1)\times (n-1)$ matrix obtained by removing row $i$ and column $j$ of $A$. The determinant of $A$ along row $i$ is recursively defined using Laplace Expansion:
\begin{equation}
    \determinate{A} = \sum_{j=1}^n A_{ij} \times \cofactor{A_{ij}} = \sum_{j=1}^n (-1)^{i+j} A_{ij}  \determinate{\tilde{A}_{ij}}
\end{equation}


\section{Properties}

\begin{theorem}
    The volume of $n$ vectors $\set{v_i}$ in $F^n$ is :
    \begin{equation}
      \absolutevaluetext{\determinate{\begin{pmatrix}
            v_1, v_2, ... v_n
        \end{pmatrix}}}
    \end{equation}
\end{theorem}


\begin{theorem}
    If a matrix has two identical rows, then $\determinate{A} = 0$.
\end{theorem}


\begin{theorem}
    \begin{equation}
        \determinate{AB} = \determinate{A} \times \determinate{B}
    \end{equation}    
\end{theorem}


\begin{theorem}[Cramer's Rule]
    For $A_{n \times n}x=b$, if $\determinate{A} \neq 0$, the solution is:
    \begin{equation}
        x_k = \frac{\determinate{M_k}}{\determinate{A}}
    \end{equation}
    where $M_k$ is obtained by replacing column $k$ of $A$ by $b$.
\end{theorem}

\begin{theorem}
    If $A$ is invertible, then let adjugate matrix be
    \begin{equation}
        \adjugate{A} = (\cofactor{A})^t
    \end{equation}
    
    Then we have
    \begin{equation}
        \inverse{A} = \frac{\adjugate{A}}{\determinate{A}}
    \end{equation}    
\end{theorem}


\begin{theorem}
  For elementary operations:
\begin{itemize}
    \item interchange two rows: $\determinate{B} = - \determinate{A}$ 
    \item multiply a row by $k$: $\determinate{B} = k \determinate{A}$
    \item add a row to another $\determinate{B} = \determinate{A}$
\end{itemize}  
\end{theorem}

\begin{theorem}
    \begin{equation}
        \determinate{A^t} = \determinate{A}
    \end{equation}    
\end{theorem}

\begin{theorem}
    If $A$ is invertible, then:
    \begin{equation}
        \determinate{\inverse{A}} = \frac{1}{\determinate{A}}
    \end{equation}    
\end{theorem}




\begin{definition}
    If $\beta$ and $\gamma$ are two ordered basis for $R^n$, they have the same orientation of and only if $\determinate{Q} > 0$, where $Q$ is change-of-coordinate matrix from $\beta$ to $\gamma$. If $\beta$ is the standard basis, then $\gamma$ is a right-handed coordinate system if $\determinate{Q} > 0$.
\end{definition}


% theory
\section{Theory}

The determinant could be viewed as the only result of special property. First we have to define two things.

\begin{definition}[n-linear function]
    $\delta$ is a n-linear function over $A$ if:
    \begin{equation}
        \delta \begin{pmatrix}
            \vdots \\
            u + kv \\
            \vdots \\
        \end{pmatrix} = \delta \begin{pmatrix}
            \vdots \\
            u \\
            \vdots \\
        \end{pmatrix} + k \times \delta \begin{pmatrix}
            \vdots \\
            v \\
            \vdots \\
        \end{pmatrix} 
    \end{equation}
\end{definition}


\begin{definition}[alternating]
    A n-linear function is alternating if $\delta (A) = 0$ when two adjacent rows of $A$ are identical.
\end{definition}

Then the determinant is the only function $\gamma$ that :
\begin{enumerate}
    \item n-linear
    \item alternating
    \item $\gamma(I_n) = 1$
\end{enumerate}

