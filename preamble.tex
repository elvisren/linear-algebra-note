\documentclass[reqno,16pt]{book}
\usepackage{amssymb}
\usepackage{latexsym}


% set section level depth and re-number it from 1
\renewcommand{\thesection}{\arabic{section}}
\setcounter{secnumdepth}{3}


% math theorem, lemma, proof
\usepackage{amsthm}
%\theoremstyle{definition}
%\newtheorem{definition}{Definition}
%\newtheorem{theorem}{Theorem}

\usepackage{thmtools}
\declaretheorem{theorem}
\declaretheorem{definition}
\declaretheorem{example}
\declaretheorem{axiom}



% for mathematical integratoin
\usepackage{commath}


% create index
\usepackage{mathtools}
\usepackage{makeidx}
\makeindex


\usepackage{listings}



% no space in itemize
\usepackage{enumitem}
\setenumerate{itemsep=0pt,partopsep=0pt,parsep=\parskip,topsep=2pt}
\setitemize{itemsep=0pt,partopsep=0pt,parsep=\parskip,topsep=2pt}
\setdescription{itemsep=0pt,partopsep=0pt,parsep=\parskip,topsep=2pt}


% set line space
\usepackage{setspace}

\usepackage[colorlinks,linkcolor=red,anchorcolor=blue,citecolor=green]{hyperref}

% set page size
\usepackage{geometry}
%\geometry{a4paper}
\geometry{a4paper,top=2cm,bottom=2cm,left=3cm,right=2cm}



% set programming code highlight
% \usepackage[chapter]{minted}


% setup bibtex
\usepackage{cite}


% include eps file
\usepackage{epsfig}



\newcommand\mathhilight[1]{\mathop{\bf #1\/}}


%linear algebra
\newcommand\nullspace[1]{\mathcal{N}(#1)}
\newcommand\rangespace[1]{\mathcal{R}(#1)}
\newcommand\absolutevalue[1]{\abs{#1}}
\newcommand\determinate[1]{\absolutevalue{#1}}
\newcommand\determinatetext[1]{\displaystyle \mathbf{det} ~\displaystyle #1}
\newcommand\coordinate[1]{\sbr{#1}}
\newcommand\projection[2]{\mathbf{proj}_{#2} #1}
\newcommand\rowvector[1]{\left[ \displaystyle #1 \right]}
\newcommand\inverse[1]{ {#1}^{-1} }

\newcommand\dimension[1]{\displaystyle \mathbf{dim}\left( #1 \right)}
\newcommand\vectorspan[1]{\displaystyle \mathbf{span}\left( #1 \right)}
\newcommand\rank[1]{\displaystyle \mathbf{rank}\left( #1 \right)}
\newcommand\innerproduct[2]{\left\langle \displaystyle #1, #2 \right\rangle}
\newcommand\trace[1]{\displaystyle \mathbf{tr}( #1 )}
\newcommand\spanset[1]{\displaystyle \mathbf{span}\left( #1 \right)}


\newcommand\adjugate[1]{\displaystyle \mathbf{adj} ~\displaystyle #1 }
\newcommand\cofactor[1]{\displaystyle \mathbf{cof} ~\displaystyle #1 }

\newcommand\columnvector[1]{\boldsymbol{#1}}
% \newcommand\norm[1]{\displaystyle \left\lVert #1 \right\rVert}

\newcommand\conjugate[1]{\overline{#1}}


% machine learning
\newcommand\subscription[2]{\boldsymbol{#1}^{(#2)}}

